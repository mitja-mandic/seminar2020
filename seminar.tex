\documentclass[a4paper]{article}
\usepackage[utf8]{inputenc}
\usepackage[T1]{fontenc}
\usepackage[slovene]{babel}
\usepackage{lmodern} 
\usepackage{hyperref}
\usepackage{blindtext}
\usepackage{amsmath}  % razna okolja za poravnane enačbe ipd.
\usepackage{amsthm}   % definicije okolij za izreke, definicije, ...
\usepackage{amssymb}  % dodatni matematični simboli

\usepackage{booktabs} % za lepše tabele
\usepackage{multirow} % za vnose v tabeli čez več vrstic

\usepackage{siunitx}

%\usepackage{xypic}

\title{Opcije na obrestne mere\\
    \textit{Interest rate options}}
\author{Mitja Mandić \\ Fakulteta za matematiko in fiziko}
\date{25.\ marec\ 2020}

\begin{document}
\begin{titlepage}
    \maketitle
    \thispagestyle{empty}
\end{titlepage}

\pagebreak
\tableofcontents
\pagebreak

\section{Uvod}
Opcijam podobne pogodbe poznamo že iz antičnih časov. Prvi znani kupec je bil grški matematik Tales iz Mileta.
Napovedano je bilo, da bo letina oljk večja od običajne, zato si je pred žetvijo zagotovil pravico uporabe večjega
števila stiskalnic olja. Ko so končali z obiranjem in je bilo oljk nad pričakovanji, je svojo opcijo izkoristil in naprej oddajal
stiskalnice po precej višji ceni kot je plačal za najem stiskalnic.

\section{Na splošno o opcijah}

Opcija je pogodba, ki kupcu (tudi nosilcu) zagotavlja določene pravice v zvezi s prodajo ali nakupom osnovnega premoženja.
O nakupni opciji (ang. \textbf{\textit{call option}}) govorimo, ko prodajalec (ang. \textit{seller, writer}) kupcu zagotovi pravico,
da bo od njega odkupil osnovno premoženje (ang. \textit{the underlying}). O prodajni pa, ko si kupec opcije zagotovi 
pravico do prodaje (ang.\textit{\textbf{put option}}).
V zameno za tveganje, ki ga spreminjanje vrednosti osnovnega premoženja prinaša, 
mora kupec plačati ceno opcije ali premijo. Ob sklenitvi pogodbe se strani dogovorita o ceni nakupa ali prodaje
(ang. \textit{strike price, exercise price}) ter datumu, do katerega je pogodbo moč izvršiti.
V tej nalogi se bomo osredotičili na opcije, kjer je osnovno premoženje obrestna mera ali obrestni instrument. 

Glede na možnost izvršitve pogodbe ločimo:
\begin{itemize}
    \item evropske opcije - izvršitev je možna le na dan zapadlosti
    \item ameriške opcije - izvršitev je možna kadarkoli do vključno dneva zapadlosti
    \item eksotične opcije - izvršitev je možna na določene datume (atlantske, bermuda, \ldots)
\end{itemize}
Ameriške opcije imajo višjo ceno kot evropske, saj ponujajo tudi več možnosti, sicer bi to pomenilo arbitražo.
Z opcijami se trguje tako na organiziranem trgu (borze), kot tudi na prostem trgu (ang. \textit{over-the-counter}).
\subsection{Lastnosti trgovanja z opcijami}
Po tem ko kupec (dolga stran) opcije plača premijo, nima več obveznosti, le še pravice. Prodajalec (kratka stran) pa se mora držati dogovora - torej prodati
oziroma kupiti premoženje, če se kupec za to odloči. Največ kar kupec lahko izgubi je premija,
prodajalec pa lahko največ toliko zasluži. Torej ima kupec možnost za profit, prodajalec pa tvega izgubo.

Pri trgovanju z opcijami lahko zavzamemo štiri pozicije:
\begin{enumerate}
    \item Dolgo prodajno (nakup prodajne opcije)
    \item Kratko prodajno (prodaja prodajne opcije)
    \item Dolgo nakupno (nakup nakupne opcije)
    \item Kratko nakupno (prodaja nakupne opcije)
\end{enumerate}
%%Oglejmo si možnost izplačil opcij, če jih izvršimo v nekem trenutku. Opcijo nosilec izvrši, če ima od tega profit, torej
%%če je notranja vrednost opcije v času izvršitve pozitivna. Vrednost opcije je za nosilca razlika med vrednostjo osnovnega premoženja in izvršilno ceno.
%%Če s $C_{t}$ označino notranjo vrednost nakupne opcije in s $P_{t}$ notranjo vrednost prodajne, je izplačilo enako:
%%\begin{center}
%%$C_{t} = max\{\text{vrednost osnovnega premoženja} - \text{premija}, 0\}$ \\
%%$P_{t} = max\{0, \text{premija} - \text{vrednost osnovnega premoženja}\}$
%%\end{center}
%Če je izplačilo pozitivno, se opcija splača (ang. \textit{in-the-money}), če je premija enaka vrednosti osnovnega premoženja
%je opcija na meji (ang. \textit{at-the-money}), in če je razlika manša od nič se opcija ne splača (ang. \textit{out-of-the-money}).
%Izplačilom in vplivom na ceno opcije se podrobneje posvetimo v poglavju \ref{cenaopcij}.

\section{Razlike med opcijami in terminskimi pogodbami}
Glavna razlika med terminskimi pogodbami in opcijami je, da opcije nosilcu ni potrebno izvršiti. Kupec ima pravico,
vendar ne obveze, kar pa ne velja za terminske pogodbe - tam morata obe strani izpolniti svoj del posla.
Posledično je razmerje med tveganjem in nagrado za instrumenta različno. Kupec terminske pogodbe pri naraščanju vrednosti
realizira \textit{dollar-for-dollar} profit in \textit{dollar-for-dollar} izgubo ko vrednost pogodbe pada, prodajalec pa ravno nasprotno. V primeru
opcij pa ni take simetrije - največ kar investitor lahko izgubi je premija, kar je hkrati največji profit, ki ga lahko realizira prodajalec;
to je kompenzacija za potencialno izgubo.

Obe strani pri terminski pogodbi morata na svoja vzdrževalna računa položiti kritje in ga vzdrževati. Takih zahtev za kupca opcije
po plačilu premije ni, saj ne glede na obnašanje osnovnega premoženja ne more izgubiti več kot premijo in zato ni potrebe po kritju.
Drugače je za prodajalca. Ker on nosi tveganje mora imeti vzdrževalni račun, nanj pa običajno položi premijo kot kritje. Če se cena 
opcije spreminja, mora kritje vzdrževati. To seveda velja le za trgovanje na borzah - na prostem trgu ni posrednika v obliki klirinške hiše.

\section{Primerjava opcij na reguliranem in prostem trgu}
Z opcijami se, tako kot z ostalimi finančnimi instrumenti, trguje na organiziranem trgu in prostem trgu. Borza, ki želi trgovati
z opcijami, mora najprej dobiti potrdila državnih regulativnih ustanov. Take opcije imajo tri glavne prednosti. Prvič, opcije na borzah
so standardizirane; imajo poenoteno izvršilno ceno in datum zapadlosti. Drugič, podobno kot pri terminskih pogodbah, ni direktne povezave
med kupcem in prodajalcem opcij. Klirinška hiša deluje kot posrednik in skrbi, da so plačila izvedena pravočasno. In tretjič, transkacijski
stroški so na borzi manjši. 

Višja cena OTC opcij odraža njihovo prilagodljivost, ki jo zahtevajo institucionalni vlagatelji ko borzna opcija ne zadostuje
njihovim ciljem. Investicijske in komercialne banke izvajajo funkcijo kupca in tudi posrednika pri trgovanju s takimi
opcijami. OTC opcija je sicer manj likvidna kot borzna, vendar to za institucionalne vlagatelje ne predstavlja večjih težav, 
saj te opcije uporabljajo kot del upravljaja s tveganjem (ang. \textit{asset/liability strategy}) in jih zadržijo do zapadlosti.

Pri borznih opcijah osnovno premoženje predstavljajo tako instrumenti s fiksnimi izplačili (obveznice),
kot terminske pogodbe o obrestni meri.

\subsection{Opcije na borzi}
V tem razdelku si bomo ogledali standardizirane opcije, s katerimi se trguje na borzi. Trguje se 
tako z opcijami na instrumente s fiksnimi denarnimi tokovi  (ang. \textit{\textbf{options on physicals}}), 
prevladujejo pa opcije na terminske pogodbe (ang. \textit{\textbf{futures options}}),
zato jim bomo posvetili večino odseka, razdelku \ref{futuresvsphysicals} pa obrazložili, zakaj prevladujejo ravno te.

\subsubsection{Opcije na terminsko pogodbo o obrestni meri}
Opcije na terminske pogodbe obstajajo na vse terminske pogodbe omenjene v seminarski nalogi Tima Kalana z naslovom 
\href{https://github.com/timkalan/seminar2020}{\textit{Terminske pogodbe o obrestni meri}}. Opcija na terminsko pogodbo o obrestni meri  daje nosilcu pravico
kupcu prodati ali od njega kupiti terminsko pogodbo po izvršilni ceni te pogodbe.

Nakupna opcija daje nosilcu pravico zavzeti dolgo pozicijo v terminski pogodbi v osnovnem premoženju. Če se opcijo odloči izvršiti,
prodajalec zavzame kratko pozicijo. Na podoben način deluje tudi prodajna opcija - če kupec izvrši opcijo, zavzame kratko pozicijo v terminski
pogodbi. 

Pojavi pa se vprašanje, kako določiti ceno terminski pogodbi. Ob izvršitvi opcije, je cena terminske pogodbe izenačena z izvršilno ceno.
Cene se nato poravnajo s tržnimi (ang. \textit{marked-to-market}) in kupec opcije bo realiziral profit. V primeru nakupne opcije, mora kupec plačati razliko med trenutno
ceno terminske pogodbe in izvršilno ceno, pri prodajni opciji pa razliko med izvršilno in trenutno ceno.

Za lažje razumevanje, si oglejmo primer. Recimo, da investitor kupi nakupno opcijo na terminsko pogodbo o obrestni meri z izvršilno ceno
\textdollar$85$. Naj bo trenutna cena te pogodbe na trgu \textdollar$95$  in naj investitor opcijo izvrši. Nosilec zavzame dolgo pozicijo v pogodbi, prodajalec
pa kratko. Ker je tržna cena pogodbe \textdollar$95$, izvršilna pa \textdollar$85$, ima investitor profit \textdollar$10$, prodajalec pa tako izgubo. Sedaj ima
investitor terminsko pogodbo s tržno vrednostjo \textdollar$95$ in jo lahko likvidira (brez stroškov likvidacije) ali pa zadrži dolgo pozicijo v
tej terminski pogodbi. Če jo likvidira, zasluži \textdollar$95$ in ima tako profit \textdollar$10$. Lahko pa zadrži pogodbo in tvega, da ji bo vrednost padla,
vendar v vsakem primeru zasluži \textdollar$10$ iz izvršitve opcije.

Oglejmo si sedaj še prodajno opcijo z izvršitveno ceno \textdollar$85$ in naj bo trenutna tržna cena pogodbe v osnovnem premoženju \textdollar$60$. Če se kupec odloči
izvršiti prodajno opcijo, zavzame kratko pozicijo v terminski pogodbi z izvršilno ceno \textdollar$85$, prodajalec opcije pa dolgo. Ker je cena
na trgu nižja kot izvršilna, prodajalec plača kupcu razliko med cenama, torej \textdollar$25$. Kupec opcije sedaj drži kratko pozicijo
v terminski pogodbi s ceno 60 in ima zopet dve možnosti: svojo pozicijo lahko likvidira, torej kupi terminsko pogodbo s ceno \textdollar$60$, ali
ohrani pozicijo. V vsakem primeru pa zasluži \textdollar$25$.

\subsubsection{Kritje}
Kupcu opcije kritja na vzdrževalni račun ni potrebno nalagati, saj po plačilu premije ne more izgubiti ničesar več. Za prodajalca pa to
ne velja. Ker je v zameno za premijo sprejel tveganje spremembe vrednosti osnovnega premoženja, mora na svoj vzdrževalni račun
naložiti kritje, ki ga mora sproti vzdrževati. To je sestavljeno iz vrednosti, ki jo zahteva terminska pogodba ter pogosto tudi premija.

\subsubsection{Opcije na terminske pogodbe vs. na fiksne denarne tokove} \label{futuresvsphysicals}
Na organiziranem trgu so opcije na terminske pogodbe o obrestni meri v večinski meri izpodrinile opcije na instrumente s fiksnim
denarnim tokom. Za to obstajajo trije razlogi: prvič, opcije na terminske pogodbe ne zahtevajo plačil natečenih obresti in jih ob izvršitvi kupcu
nakupne ali prodajalcu prodajne opcije ni potrebno izplačati.

Drugič, stran, ki mora instrument izročiti, tvega, da bo ob izvršitvi opcije na trgu pomanjkanje instrumenta, kar mu dvigne ceno (ang. \textit{delivery squeeze}).
Ker je za trenutne terminske pogodbe na obrestno mero zaloga praktično neskončna, ni strahu pred naraščanjem cen. 

In tretjič, če želimo določiti ceno opciji, je pomembno poznati vrednost osnovnega premoženja.
Trenutne cene obveznic je običajno težje dobiti kot trenutne cene terminskih pogodb, saj se z obveznicami v veliki meri trguje na prostem trgu, kjer
se cen ne poroča osrednji bazi podatkov. Investitor, ki želi kupiti opcijo na državno obveznico bi torej moral kontaktirati več trgovcev, 
da bi si ustvaril jasnejšo sliko o njeni ceni ter določil ceno opcije. Nasprotno pa se s terminskimi pogodbami večinoma trguje na organiziranem trgu in so zato vsi 
podatki o cenah zbrani na enem mestu. 

\subsection{Opcije na prostem trgu}
Investitorji, ki želijo kupiti opcijo na določeno državno obveznico, Ginnie Mae obveznico (ang. \textit{mortgage-based security} oziroma širše \textit{pass-through security};
obveznica, ki jo financirajo posojilojemalci skozi posrednike, ki odplačujejo kredite za hiše, avtomobile, \ldots), lahko take opcije dobijo na prostem trgu.
OTC opcije (ang. tudi \textit{dealer options}) se uporabljajo za ščitenje pred specifičnim tveganjem. Običajno se obdobje do zapadlosti opcije prekriva z obdobjem 
ko investitor potrebuje zaščito in ga zato likvidnost opcije ne skrbi.

V trgovanju z OTC opcijami med kupcem in prodajalcem ni posrednika v obliki klirinške hiše. V primeru terminskih pogodb, sta tako
obe strani izpostavljeni tveganju, da nasprotna stran ne bo mogla izplačati obveznosti (ang. \textit{counterparty risk}). Pri opcijah
pa je (po plačani premiji) tveganju, da prodajalec ne bo mogel izplačati obveznosti, izpostavljen le kupec.

Omejitev pri OTC opcijah ni - če se prodajalec lahko zaščiti pred tveganjem, ki ga prinaša potencialna izvršitev opcije, jo bo prodal.
Tako te vrste opcije tudi niso omejene na ameriške ali evropske - pojavljajo se tudi modificirane ameriške, bermuda, \ldots.

\section{Cena opcij} \label{cenaopcij}
Na ceno opcij poglavitno vplivata dva faktorja: notranja vrednost (ang. \textit{intrinsic value}) in časovna vrednost(ang. \textit{time value}).

\textbf{Notranja vrednost} je zaslužek, ki ga realiziramo, če opcijo v danem trenutku izvršimo.
Za nakupno opcijo je notranja vrednost pozitivna, če je razlika med tržno vrednostjo osnovnega premoženja in izvršilno ceno večja od nič. Za prodajno
pa razlika med izvršilno ceno in vrednostjo osnovnega premoženja.
\begin{center}
    $\text{Nakupna opcija} = max\{\text{vrednost osnovnega premoženja} - \text{izvršilna cena}, 0\}$ \\
    $\text{Prodajna opcija} = max\{0, \text{izvršilna cena} - \text{vrednost osnovnega premoženja}\}$
\end{center}
Če je notranja vrednost opcije večja od nič, pravimo, da se opcija splača (ang. \textit{in-the-money}), če je enaka nič, je opcija na meji (ang. \textit{at-the-money}),
in če je manjša od nič, se ne splača (ang. \textit{out-of-the-money}).
Na primer, če je izvršilna cena nakupne opcije \textdollar$100$ in je tržna vrednost osnovnega premoženja \textdollar$105$, je notranja vrednost \textdollar$5$. 
Če torej nosilec opcije proda osnovno premoženje za \textdollar$105$ in nato kupi opcijo za \textdollar$100$ realizira profit \textdollar$5$.
Oglejmo si še primer s prodajno opcijo. Če je njena izvršilna cena \textdollar$100$ in trenutna tržna vrednost osnovnega premoženja \textdollar$92$, je notranja 
vrednost take opcije \textdollar$8$. Nosilec take opcije realizira tolikšen profit, če vzporedno izvrši opcijo (proda premoženje prodajalcu za \textdollar$100$) in
kupi to premoženje na trgu za \textdollar$92$.

\begin{table}[h]
    \centering
    \begin{tabular} { c | c | c }
        \toprule
        & Nakupna opcija & Prodajna opcija \\
        \hline
        \textbf{Tržna cena > izvršilna cena} & \textbf{Tržna - Izvršilna} & \textbf{0}  \\
        \hline
        & opcija se splača & opcija se ne splača \\ 
        & \textit{in-the-money} & \textit{out-of-the-money} \\
        \hline
        \textbf{Tržna cena < izvršilna cena} &\textbf{0} & \textbf{Izvršilna - Tržna} \\
        \hline
        & opcija se ne splača & opcija se splača \\
        & \textit{out-of-the-money} &\textit{in-the-money}\\ 
        \hline
        \textbf{Tržna cena = izvršilna cena} & \textbf{0} & \textbf{0} \\
        \hline
        & opcija je na meji & opcija je na meji \\
        & \textit{at-the-money} &  \textit{at-the-money} \\  
        \bottomrule
    \end{tabular}
    \caption{Pregled izrazov}
\end{table}

\textbf{Časovna vrednost} je razlika med ceno opcije in njeno notranjo vrednostjo. Kupec opcije upa, da bodo spremembe na trgu pred
dnevom zapadlosti spremenile vrednost osnovnega premoženja in tako povečale potencialni profit. Zaradi tega je kupec pripravljen plačati
premijo za opcijo. 

Nakupna opcija, ki ima izvršilno ceno 100$\$$ in ceno 9$\$$ z osnovnim premoženjem v vrednosti 105$\$$, ima časovno vrednost 4$\$$,
saj ima notranjo vrednost 5$\$$. Če pa ima opcija izvršilno ceno 90$\$$, je njena časovna vrednost 9$\$$, saj je notranja vrednost enaka nič.

\subsection{Vplivi na ceno opcije}
Na ceno opcije vpliva 6 dejavnikov:
\begin{enumerate}
    \item trenutna tržna cena osnovnega premoženja
    \item izvršilna cena opcije
    \item dolžina obdobja do zapadlosti
    \item pričakovane spremembe v obrestnih merah do zapadlosti opcije
    \item kratkoročne spremembe netveganih obrestnih mer
    \item kuponska izplačila do zapadlosti opcije
\end{enumerate}
V sledečem razdelku bomo obravnavali zgoraj navedene vplive na ceno opcije, pri tem pa predpostavili, da so vsi ostali
dejavniki nespremenjeni.

\textbf{Trenutna tržna cena osnovnega premoženja} vpliva na ceno opcije. V primeru nakupne opcije, se ob 
dvigu cene osnovnega premoženja dvigne tudi cena opcije. Razlog za to je dvig notranje vrednosti,
 ki raste z dvigom cen osnovnega premoženja. Obratno velja za prodajne
opcije - ob dvigu cene osnovnega premoženja, pade notranja vrednost in posledično tudi cena opcije.

Nižja kot je \textbf{izvršilna cena opcije}, višja je cena nakupne opcije. Za prodajne opcije pa velja: višja kot je cena 
osnovnega premoženja, višja je cena opcije.

Daljši kot je \textbf{čas do zapadlosti}, višja je cena opcije. Za opcijo velja, da ko preteče datum zapadlosti, 
izgubi svojo vrednost.
Bliže kot smo temu datumu, manj je časa, da vrednost osnovnega premoženja zraste (za kupca prodajne opcije) in pade (za kupca nakupne) ---
za kompenzacijo časovne vrednosti kupcu --- in zato verjetnost željene spremembe cene pade. Posledično se cena ameriške opcije bliža njeni
notranji vrednosti, ko se bližamo dnevu zapadlosti.

Večje kot so \textbf{pričakovane spremembe obrestnih mer}, več prodajalec zahteva za opcijo in tudi investitor je pripravljen
zanjo plačati več. To je, ker z večjo volatilnostjo obrestnih mer raste tudi verjetnost, da se bo cena osnovnega premoženja premaknila
v smer, ki jo želi kupec. Metode za oceno sprememb obrestnih mer pa je v svoji nalogi predstavila Anja Trobec.

Oglejmo si še spremembe v \textbf{kratkoročnih netveganih obrestnih merah}. Nakup opcije na osnovno premoženje 
je za investitorja privlačnejše, kot nakup osnovnega premoženja, saj je tako razlika med ceno 
premoženja in ceno opcije pri netvegani obrestni meri na voljo za ponovno investicijo. Višja kot je kratkoročna netvegana 
obrestna mera, višja je cena osnovnega premoženja in zato nakupna opcija privlačnejša. Posledično je tudi cena nakupne opcije, pri 
višji kratkotrajni netvegani obrestni meri višja. V primeru prodajne opcije, pa je alternativa nakupu take opcije kratka prodaja
osnovnega premoženja. Profit iz kratke prodaje lahko nato znova investiramo po kratkoročni netvegani obrestni meri. Ko se le-ta poveča,
je kratka prodaja osnovnega premoženja privlačnejša kot nakup prodajne opcije, zato cena te pade.

\textbf{Kuponska izplačila} osnovnega premoženja nižajo ceno nakupni opciji saj je privlačneje imeti to premoženje, kot nakupno 
opcijo nanj (v primeru nakupne opcije, kuponov ne prejemamo). Višja kot so kuponska izplačila osnovnega premoženja, nižja je cena
nakupne opcije nanj. Nasprotno pa velja za prodajne opcije. Nakup take opcije primerjamo s kratko prodajo osnovnega premoženja;
višji kot so kuponi, več moramo izplačevati. Višji kot so kuponi, bolj privlačno je kupiti prodajno opcijo kot kratko prodati osnovno
premoženje in zato cena prodajni opciji naraste.

\begin{table}
    \begin{tabular}{l|c|c}
        Dejavnik & Nakupna opcija & Prodajna opcija \\
        \hline\hline
        Cena osnovnega premoženja & poveča & zmanjša \\
        Izvršilna cena & zmanjša & poveča \\
        Čas do zapadlosti & poveča & poveča \\
        Pričakovane spremembe obrestnih mer & poveča & poveča \\
        Kratkoročne netvegane obrestne mere & poveča & zmanjša \\
        Kuponi & zmanjša & poveča
    \end{tabular}
    \caption{Vplivi razičnih dejavnikov na ceno opcije}
\end{table}

\subsection{Cena opcije na terminsko pogodbo}
Cena opcije na terminsko pogodbo je izražena v 64-inah procenta nominalne cene terminske pogodbe. Torej cena 24 pomeni 24/64
1\% nominalne cene. Ker je nominalna cena terminske pogodbe na državno obveznico 100.000\textdollar, cena 24 pomeni:
\(\frac{24/64}{100} \times 100.000\$ = 375\$\). V splošnem je cena opcije na terminsko pogodbo, ki kotira kot \textit{Q}:
\begin{center}
    Cena opcije $ = \frac{\textit{\text{Q}}/64}{100} \times 100.000\$$
\end{center}
\end{document}