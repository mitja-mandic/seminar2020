\documentclass[a4paper]{article}
\usepackage[utf8]{inputenc}
\usepackage[T1]{fontenc}
\usepackage[slovene]{babel}
\usepackage{lmodern} 
\usepackage{hyperref}
\usepackage{blindtext}

\title{Opcije na obrestne mere\\
    \textit{Interest rate options}}
\author{Mitja Mandić \\ Fakulteta za matematiko in fiziko}
\date{25.\ marec\ 2020}

\begin{document}

\maketitle

\pagebreak

\begin{abstract}

\end{abstract}
\pagebreak

\section{Uvod}
Opcijam podobne pogodbe poznamo že iz antičnih časov. Prvi znani kupec je bil grški matematik Tales iz Mileta.
Napovedano je bilo, da bo letina oljk večja od običajne, zato si je pred žetvijo zagotovil pravico uporabe večjega
števila stiskalnic olja. Ko so končali z obiranjem in je bilo oljk nad pričakovanji, je svojo opcijo izkoristil in naprej oddajal
stiskalnice po precej višji ceni kot je plačal za najem stiskalnic.

\section{Na splošno o opcijah}
Opcija je pogodba, v kateri prodajalec (ang \textit{writer}) kupcu zagotovi pravico, vendar ne obveze, da bo prodajalcu prodal
ali od njega kupil osnovno premoženje (ang \textit{the underlying}) do vključno vnaprej dogovorjenega datuma (\textit{expiration date}). Po tem opcija zapade.
Prodajalec v zameno dobi določeno vsoto denarja, ki jo imenujemo premija, cena po kateri se premoženje predaja oziroma kupuje
pa je izvršilna cena (ang \textit{strike price}). Glede na možnost izvršitve pogodbe, ločimo:
\begin{itemize}
    \item evropske opcije - izvršitev je možna le na dan zapadlosti
    \item ameriške opcije - izvršitev je možna kadarkoli do vključno dneva zapadlosti
    \item eksotične opcije - izvršitev je možna na določene datume (atlantske, bermuda, \ldots)
\end{itemize}
Z opcijami se trguje tako na organiziranem trgu (borze), kot tudi na prostem trgu (ang \textit{over-the-counter}). Na borzah so
standardizirane in se trguje predvsem z opcijami na delnice, terminske pogodbe in blago, medtem ko se na prostem trgu trguje z opcijami
na devizne tečaje in obrestne mere.


\subsection{Nakupne opcije}
Nakupne opcije (ang \textit{call options}) dajejo kupcu pravico, ne pa tudi obveze, da bo do zapadlosti kupil premoženje od prodajalca.
Slednji je zavezan prodati, če se kupec za to odloči. 

\end{document}