\documentclass[a4paper]{article}
\usepackage[utf8]{inputenc}
\usepackage[T1]{fontenc}
\usepackage[slovene]{babel}
\usepackage{lmodern} 
\usepackage{hyperref}
\usepackage{blindtext}
\usepackage{amsmath}  % razna okolja za poravnane enačbe ipd.
\usepackage{amsthm}   % definicije okolij za izreke, definicije, ...
\usepackage{amssymb}  % dodatni matematični simboli

%\usepackage{xypic}

\title{Opcije na obrestne mere\\
    \textit{Interest rate options}}
\author{Mitja Mandić \\ Fakulteta za matematiko in fiziko}
\date{25.\ marec\ 2020}

\begin{document}

\maketitle

\pagebreak

\begin{abstract}

\end{abstract}
\pagebreak
\tableofcontents
\pagebreak

\section{Uvod}
Opcijam podobne pogodbe poznamo že iz antičnih časov. Prvi znani kupec je bil grški matematik Tales iz Mileta.
Napovedano je bilo, da bo letina oljk večja od običajne, zato si je pred žetvijo zagotovil pravico uporabe večjega
števila stiskalnic olja. Ko so končali z obiranjem in je bilo oljk nad pričakovanji, je svojo opcijo izkoristil in naprej oddajal
stiskalnice po precej višji ceni kot je plačal za najem stiskalnic.

\section{Na splošno o opcijah}

Opcija je pogodba, v kateri prodajalec (ang. \textit{writer}) kupcu (tudi nosilcu) zagotovi pravico, 
vendar ne obveze, da bo od nosilca kupil (ang. \textit{put option})
ali mu prodal (ang. \textit{call option}) osnovno premoženje (ang. \textit{the underlying}) do 
vključno vnaprej dogovorjenega datuma (\textit{expiration date}).
Po tem opcija zapade.  V tej nalogi se bomo osredotičili na opcije, kjer je
osnovno premoženje obrestna mera ali obrestni instrument. Prodajalec v zameno dobi določeno vsoto denarja, ki jo imenujemo premija, 
cena po kateri se premoženje predaja oziroma kupuje pa je izvršilna cena (ang. \textit{strike price}).
Glede na možnost izvršitve pogodbe ločimo:
\begin{itemize}
    \item evropske opcije - izvršitev je možna le na dan zapadlosti
    \item ameriške opcije - izvršitev je možna kadarkoli do vključno dneva zapadlosti
    \item eksotične opcije - izvršitev je možna na določene datume (atlantske, bermuda, \ldots)
\end{itemize}

Ameriške opcije imajo višjo ceno kot evropske, saj ponujajo tudi več možnosti. Drugače bi to pomenilo arbitražo.
Z opcijami se trguje tako na organiziranem trgu (borze), kot tudi na prostem trgu (ang. \textit{over-the-counter}). Na borzah so
standardizirane in se trguje predvsem z opcijami na delnice, terminske pogodbe in blago, medtem ko se na prostem trgu trguje z opcijami
na devizne tečaje in obrestne mere.

\subsection{Lastnosti trgovanja z opcijami}
Po tem ko kupec opcije plača premijo, nima več obveznosti, le še pravice, prodajalec pa se mora držati dogovora - torej prodati
oziroma kupiti premoženje, če se kupec za to odloči. Največ kar investitor lahko izgubi je premija,
prodajalec pa lahko največ zasluži s prodajo opcije. Torej ima investitor možnost za potencialen profit,
prodajalec pa potencialno tveganje za izgubo. \par
Pri trgovanju z opcijami lahko zavzamemo štiri pozicije:
\begin{enumerate}
    \item Dolgo prodajno (nakup prodajne opcije)
    \item Kratko prodajno (prodaja prodajne opcije)
    \item Dolgo nakupno (nakup nakupne opcije)
    \item Kratko nakupno (prodaja nakupne opcije)
\end{enumerate}
\pagebreak
Oglejmo si možnost izplačil opcij, če jih izvršimo na dan zapadlosti. Opcijo nosilec izvrši, če ima od tega profit, torej
če je vrednost opcije v času izvršitve pozitivna. vrednost opcije je za nosilca razlika med ceno osnovnega premoženja in premijo.
Če s $C_{t}$ označino vrednost nakupne opcije in s $P_{t}$ vrednost prodajne, je izplačilo enako:
\begin{center}
$C_{t}$ = max$\{$vrednost osnovnega premoženja - premija, 0$\}$ \\
$P_{t}$ = max$\{$0, premija - vrednost osnovnega premoženja$\}$
\end{center}
Če je izplačilo pozitivno, se opcija splača (ang. \textit{in the money}), če je premija enaka vrednosti osnovnega premoženja
je opcija na meji (ang. \textit{at the money}), in če je razlika manša od nič se opcija ne splača(ang. \textit{out of money}).

\section{Razlike med opcijami in terminskimi pogodbami}
Glavna razlika med terminskimi pogodbami in opcijami je, da opcije nosilcu ni potrebno ni potrebno izvršiti. Kupec ima pravico,
vendar ne obveze, kar pa ne velja za terminske pogodbe - tam morata obe strani izpolniti svoj del posla.
Posledično je razmerje med tveganjem in nagrado za instrumenta različno. Kupec terminske pogodbe pri naraščanju vrednosti
realizira dolar-za-dolar profit in dolar-za-dolar izgubo ko vrednost pogodbe pada, prodajalec pa ravno nasprotno. V primeru
opcij pa ni take simetrije - največ kar investitor lahko izgubi je premija, kar je hkrati največji profit, ki ga lahko dobi prodajalec;
to je kompenzacija za potencialno izgubo.

Obe stranki pri terminski pogodbi morata na svoja vzdrževalna računa položiti kritje in ga vzdrževati. Takih zahtev za kupca opcije
po plačilu premije ni, saj ne glede na obnašanje osnovnega premoženja ne more izgubiti več kot premijo in zato ni potrebe po kritju.
Drugače je za prodajalca. Ker on nosi tveganje mora imeti vzdrževalni račun, nanj pa običajno položi premijo kot kritje. Če se cena 
opcije spreminja, mora kritje vzdrževati.

\section{Primerjava opcij na reguliranem in prostem trgu}
Z opcijami se, tako kot z ostalimi finančnimi instrumenti, trguje na organiziranem trgu in prostem trgu. Borza, ki želi trgovati
z opcijami, mora najprej dobiti potrdila državnih regulativnih ustanov. Take opcije imajo tri glavne prednosti. Prvič, opcije na borzah
so standardizirane; imajo poenoteno izvršilno ceno in datum zapadlosti. Drugič, podobno kot pri terminskih pogodbah, ni direktne povezave
med kupcem in prodajalcem opcij. Klirinška hiša deluje kot posrednik in skrbi, da so plačila izvedena pravočasno. In tretjič, transkacijski
stroški so na borzi manjši. 

Višja cena OTC opcij odraža njihovo prilagodljivost, ki jo zahtevajo institucionalni vlagatelji ko borzna opcija ne zadostuje
njihovim investicijskim ciljem. Investicijske in komercialne banke izvajajo funkcijo kupca in tudi posrednika pri trgovanju s takimi
opcijami. OTC opcija je sicer manj likvidna kot borzna, vendar to za institucionalne vlagatelje ne predstavlja večjih težav, 
saj te opcije uporabljajo kot del upravljaja s tveganjem (ang. \textit{asset/liability strategy}) in jih zadržijo do zapadlosti.

Pri borznih opcijah osnovni kapital lahko predstavljajo tako instrumenti s fiksno obrestno mero (ang. \textit{options on physicals}),
kot terminske pogodbe o obrestni meri.

\subsection{Opcije na borzi}
V tem razdelku si bomo ogledali standardizirane opcije, s katerimi se trguje na borzi. Prevladujejo opcije na terminske pogodbe,
zato bomo njim posvetili večino odseka, na koncu pa obrazložili, zakaj prevladujejo ravno te.

\subsubsection{Opcije na terminsko pogodbo o obrestni meri}
Opcije na terminske pogodbe obstajajo na vse terminske pogodbe omenjene v seminarski nalogi Tima Kalana z naslovom \textit{Terminske
pogodbe o obrestni meri}. Opcija na terminsko pogodbo o obrestni meri (ang. \textit{\textbf{futures options}}) daje nosilcu pravico
kupcu prodati ali od njega kupiti terminsko pogodbo po izvršilni ceni te pogodbe.

Nakupna opcija daje nosilcu pravico zavzeti dolgo pozicijo v terminski pogodbi v osnovnem premoženju. Če se opcijo odloči izvršiti,
kupec zavzame kratko pozicijo. Na podoben način deluje tudi prodajna opcija - če kupec izvrši opcijo, zavzame kratko pozicijo v terminski
pogodbi. 

Pojavi pa se vprašanje, kako določiti ceno terminski pogodbi. Ob izvršitvi opcije, je cena terminske pogodbe izenačena z izvršilno ceno.
Cene se nato poravnajo s tržnimi in kupec opcije bo realiziral profit. V primeru nakupne opcije, mora kupec plačati razliko med trenutno
ceno terminske pogodbe in izvršilno ceno, pri prodajni opciji pa razliko med izvršilno in trenutno ceno.

Za lažje razumevanje, si oglejmo primer. Recimo, da investitor kupi nakupno opcijo na terminsko pogodbo o obrestni meri z izvršilno ceno
85$\$$. Naj bo trenutna cena te pogodbe na trgu 95$\$$ in naj investitor opcijo izvrši. Nosilec zavzame dolgo pozicijo v pogodbi, prodajalec
pa kratko. Ker je tržna cena pogodbe 95$\$$, izvršilna pa 85$\$$, ima investitor profit 10$\$$, prodajalec pa tako izgubo. Sedaj ima
investitor terminsko pogodbo s tržno vrednostjo 95$\$$ in jo lahko likvidira (brez stroškov likvidacije) ali pa zadrži dolgo pozicijo v
tej terminski pogodbi. Če jo likvidira, zasluži 95$\$$ in ima tako profit 10. Lahko pa zadrži pogodbo in tvega, da ji bo vrednost padla,
vendar v vsakem primeru zasluži 10$\$$ iz izvršitve opcije.

Oglejmo si sedaj še prodajno opcijo z izvršitveno ceno 85$\$$ in naj bo trenutna tržna cena pogodbe 60$\$$. Če se kupec odloči
izvršiti prodajno opcijo, zavzame kratko pozicijo v terminski pogodbi z izvršilno ceno 85$\$$, prodajalec opcije pa dolgo. Ker je cena
na trgu nižja kot izvršilna, prodajalec plača kupcu razliko med tema cenama, torej 25$\$$. Kupec opcije sedaj drži kratko pozicijo
v terminski pogodbi s ceno 60 in ima zopet dve možnosti: svojo pozicijo lahko likvidira, torej kupi terminsko pogodbo s ceno 60$\$$, ali
ohrani pozicijo. V vsakem primeru pa zasluži 25$\$$.

\subsubsection{Kritje}
Kupcu opcije kritja na vzdrževalni račun ni potrebno nalagati, saj po plačilu cene opcije ne more izgubiti ničesar več. Enako 
pa ne velja za prodajalca; ker se je odločil nositi celotno tveganje spreminjanja obrestnih mer, mora na svoj vzdrževalni račun
naložiti kritje, ki ga od njega zahteva terminska pogodba, na katero je opcija napisana, ter to kritje obnavljati po potrebi.
Dodatno mora naložiti tudi ceno, ki jo je prejel za opcijo.

\section{Opcije na prostem trgu}


\section{Cena opcije}
Na ceno opcije poglavitno vplivata dva faktorja: osnovna vrednost (ang. \textit{intrinsic value}) in časovna vrednost.

\textbf{Osnovna vrednost} je ekonomska vrednost, če opcijo takoj izvršimo. Ta je enaka nič, če je ekonomska vrednost manjša ali enaka nič.
Za nakupno opcijo je osnovna vrednost pozitivna, če je razlika med tržno vrednostjo osnovnega premoženja in izvršilno ceno večja od nič.

Na primer, če je cena izvršilna cena opcije 100$\$$ in je tržna vrednost osnovnega premoženja 105$\$$, je osnovna vrednost 5$\$$. 
Če torej nosilec opcije proda osnovno premoženje za 105$\$$ in nato kupi opcijo za 100$\$$ realizira profit 5$\$$.

\textbf{Časovna vrednost} je razlika med ceno opcije in njeno osnovno vrednostjo. Kupec opcije upa, da bodo spremembe na trgu pred
dnevom zapadlosti spremenile vrednost osnovnega premoženja in tako povečale potencialni profit. Zaradi tega je kupec pripravljen plačati
premijo za opcijo. 

Nakupna opcija, ki ima izvršilno ceno 100$\$$ in ceno 9$\$$ z osnovnim premoženjem v vrednosti 105$\$$, ima časovno vrednost 4$\$$,
saj ima osnovno ceno 5$\$$. Če pa ima opcija izvršilno ceno 90$\$$, je njena časovna vrednost 9$\$$, saj je osnovna vrednost enaka nič.


\end{document}