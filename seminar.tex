\documentclass[a4paper]{article}
\usepackage[utf8]{inputenc}
\usepackage[T1]{fontenc}
\usepackage[slovene]{babel}
\usepackage{lmodern} 
\usepackage{hyperref}
\usepackage{blindtext}
\usepackage{amsmath}  % razna okolja za poravnane enačbe ipd.
\usepackage{amsthm}   % definicije okolij za izreke, definicije, ...
\usepackage{amssymb}  % dodatni matematični simboli
%\usepackage{xypic}

\title{Opcije na obrestne mere\\
    \textit{Interest rate options}}
\author{Mitja Mandić \\ Fakulteta za matematiko in fiziko}
\date{25.\ marec\ 2020}

\begin{document}

\maketitle

\pagebreak

\begin{abstract}

\end{abstract}
\pagebreak
\tableofcontents
\pagebreak

\section{Uvod}
Opcijam podobne pogodbe poznamo že iz antičnih časov. Prvi znani kupec je bil grški matematik Tales iz Mileta.
Napovedano je bilo, da bo letina oljk večja od običajne, zato si je pred žetvijo zagotovil pravico uporabe večjega
števila stiskalnic olja. Ko so končali z obiranjem in je bilo oljk nad pričakovanji, je svojo opcijo izkoristil in naprej oddajal
stiskalnice po precej višji ceni kot je plačal za najem stiskalnic.

\section{Na splošno o opcijah}

Opcija je pogodba, v kateri prodajalec (ang. \textit{writer}) kupcu (tudi nosilcu) zagotovi pravico, 
vendar ne obveze, da bo od nosilca kupil (ang. \textit{put option})
ali mu prodal (ang. \textit{call option}) osnovno premoženje (ang. \textit{the underlying}) do 
vključno vnaprej dogovorjenega datuma (\textit{expiration date}).
Po tem opcija zapade.  V tej nalogi se bomo osredotičili na opcije, kjer je
osnovno premoženje obrestna mera ali obrestni instrument. Prodajalec v zameno dobi določeno vsoto denarja, ki jo imenujemo premija, 
cena po kateri se premoženje predaja oziroma kupuje pa je izvršilna cena (ang. \textit{strike price}).
Glede na možnost izvršitve pogodbe ločimo:
\begin{itemize}
    \item evropske opcije - izvršitev je možna le na dan zapadlosti
    \item ameriške opcije - izvršitev je možna kadarkoli do vključno dneva zapadlosti
    \item eksotične opcije - izvršitev je možna na določene datume (atlantske, bermuda, \ldots)
\end{itemize}

Ameriške opcije imajo višjo ceno kot evropske, saj ponujajo tudi več možnosti. Drugače bi to pomenilo arbitražo.
Z opcijami se trguje tako na organiziranem trgu (borze), kot tudi na prostem trgu (ang. \textit{over-the-counter}). Na borzah so
standardizirane in se trguje predvsem z opcijami na delnice, terminske pogodbe in blago, medtem ko se na prostem trgu trguje z opcijami
na devizne tečaje in obrestne mere.

\subsection{Lastnosti trgovanja z opcijami}
Po tem ko kupec opcije plača premijo, nima več obveznosti, le še pravice, prodajalec pa se mora držati dogovora - torej prodati
oziroma kupiti premoženje, če se kupec za to odloči. Največ kar investitor lahko izgubi je premija,
prodajalec pa lahko največ zasluži s prodajo opcije. Torej ima investitor možnost za potencialen profit,
prodajalec pa potencialno tveganje za izgubo. \par
Pri trgovanju z opcijami lahko zavzamemo štiri pozicije:
\begin{enumerate}
    \item Dolgo prodajno (nakup prodajne opcije)
    \item Kratko prodajno (prodaja prodajne opcije)
    \item Dolgo nakupno (nakup nakupne opcije)
    \item Kratko nakupno (prodaja nakupne opcije)
\end{enumerate}
\pagebreak
Oglejmo si možnost izplačil opcij, če jih izvršimo na dan zapadlosti. Opcijo nosilec izvrši, če ima od tega profit, torej
če je vrednost opcije v času izvršitve pozitivna. vrednost opcije je za nosilca razlika med ceno osnovnega premoženja in premijo.
Če s $C_{t}$ označino vrednost nakupne opcije in s $P_{t}$ vrednost prodajne, je izplačilo enako:
\begin{center}
$C_{t}$ = max$\{$vrednost osnovnega premoženja - premija, 0$\}$ \\
$P_{t}$ = max$\{$0, premija - vrednost osnovnega premoženja$\}$
\end{center}
Če je izplačilo pozitivno, se opcija splača (ang. \textit{in the money}), če je premija enaka vrednosti osnovnega premoženja
je opcija na meji (ang. \textit{at the money}), in če je razlika manša od nič se opcija ne splača(ang. \textit{out of money}).

\section{Razlike med opcijami in terminskimi pogodbami}
Glavna razlika med terminskimi pogodbami in opcijami je, da opcije nosilcu ni potrebno ni potrebno izvršiti. Kupec ima pravico,
vendar ne obveze, kar pa ne velja za terminske pogodbe - tam morata obe strani izpolniti svoj del posla.
Posledično je razmerje med tveganjem in nagrado za instrumenta različno. Kupec terminske pogodbe pri naraščanju vrednosti
realizira dolar-za-dolar profit in dolar-za-dolar izgubo ko vrednost pogodbe pada, prodajalec pa ravno nasprotno. V primeru
opcij pa ni take simetrije - največ kar investitor lahko izgubi je premija, kar je hkrati največji profit, ki ga lahko dobi prodajalec;
to je kompenzacija za potencialno izgubo.

Obe stranki pri terminski pogodbi morata na svoja vzdrževalna računa položiti kritje in ga vzdrževati. Takih zahtev za kupca opcije
po plačilu premije ni, saj ne glede na obnašanje osnovnega premoženja ne more izgubiti več kot premijo in zato ni potrebe po kritju.
Drugače je za prodajalca. Ker on nosi tveganje mora imeti vzdrževalni račun, nanj pa običajno položi premijo kot kritje. Če se cena 
opcije spreminja, mora kritje vzdrževati.

\section{Primerjava opcij na reguliranem in prostem trgu}
Z opcijami se, tako kot z ostalimi finančnimi instrumenti, trguje na organiziranem trgu in prostem trgu. Borza, ki želi trgovati
z opcijami, mora najprej dobiti potrdila državnih regulativnih ustanov. Take opcije imajo tri glavne prednosti. Prvič, opcije na borzah
so standardizirane; imajo poenoteno izvršilno ceno in datum zapadlosti. Drugič, podobno kot pri terminskih pogodbah, ni direktne povezave
med kupcem in prodajalcem opcij. Klirinška hiša deluje kot posrednik in skrbi, da so plačila izvedena pravočasno. In tretjič, transkacijski
stroški so na borzi manjši. 

Višja cena OTC opcij odraža njihovo prilagodljivost, ki jo zahtevajo institucionalni vlagatelji ko borzna opcija ne zadostuje
njihovim investicijskim ciljem. Investicijske in komercialne banke izvajajo funkcijo kupca in tudi posrednika pri trgovanju s takimi
opcijami. OTC opcija je sicer manj likvidna kot borzna, vendar to za institucionalne vlagatelje ne predstavlja večjih težav, 
saj te opcije uporabljajo kot del upravljaja s tveganjem (ang. \textit{asset/liability strategy}) in jih zadržijo do zapadlosti.

Pri borznih opcijah osnovni kapital lahko predstavljajo tako instrumenti s fiksno obrestno mero (ang. \textit{options on physicals}),
kot terminske pogodbe o obrestni meri.

\subsection{Opcije na borzah}

\subsubsection{Opcije na terminsko pogodbo o obrestni meri}


\end{document}