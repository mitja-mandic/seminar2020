\documentclass[14pt]{beamer}
\usepackage[T1]{fontenc}
\usepackage[utf8]{inputenc}
\usepackage[slovene]{babel}
\usepackage{pgfpages}
\mode<presentation>
\usepackage{palatino}
%\usetheme{Goettingen}
\usefonttheme{serif}
\setbeamertemplate{navigation symbols}{}
\setbeamertemplate{footline}[frame number]{}
\setbeamertemplate{itemize items}[circle]


\begin{document}
\title{Opcije na obrestne mere \\ \textit{Interest Rate Options}}
\author{Mitja Mandić}
\date{21. april 2020}

\begin{frame}
    \titlepage
\end{frame}

\begin{frame}
    \frametitle{Uvod}
    Opcijam podobne pogodbe poznamo že iz antike
\end{frame}

\begin{frame}
    \frametitle{Na splošno o opcijah}
    \begin{itemize}
        \item Opcije so izveden finančni instrument, kjer kupec dobi pravico, da od prodajalca opcije kupi 
        ali mu proda osnovno premoženje
        \item Glede na možnost izvedbe poznamo tri osnovne skupine opcij:
        \begin{itemize}
            \item evropske
            \pause
            \item ameriške
            \pause
            \item modificirane ameriške
        \end{itemize}
    \end{itemize}
\end{frame}

\begin{frame}
    \frametitle{Trgovanje z opcijami}
    \begin{itemize}
        \item Ko nosilec plača premijo prodajalcu, nima več obveznosti, le še pravice
        \item Nosilec lahko izgubi največ premijo, prodajalec pa lahko največ toliko profitira
    \end{itemize}
\end{frame}

\begin{frame}
    \frametitle{Trgovanje z opcijami}
    V trgovanju z opcijami lahko zavzamemo štiri pozicije:
    \begin{itemize}
        \item Dolgo nakupno
        \item Kratko nakupno
        \item Dolgo prodajno
        \item Kratko prodajno
    \end{itemize}
\end{frame}

\begin{frame}
    \frametitle{Razlika med opcijami in terminskimi pogodbami}
    \begin{itemize}
        \item Nosilcu opcije po plačilu premije ni potrebno več reagirati
        \item Prodajalec opcije pa se mora ravnati po željah kupca - tu je kupec izpostavljen tveganju
        \pause
        \item Drugačna shema tveganja in zaslužka
        \pause
        \item Za nosilca ni potrebe po kritju, prodajalec pa kritje mora vzdrževati
    \end{itemize}
\end{frame}

\begin{frame}
    \frametitle{Opcije na reguliranem in prostem trgu}
    \begin{itemize}
        \item Opcije na borzi so standardizirane, posrednik pri trgovanju je klirinška hiša, transakcijski stroški so nižji
        \item Cena opcij na borzah je nižja kot na prostem trgu, vendar pa so OTC opcije bolj prilagodljive
        \item Na borzah prevladujejo opcije na terminsko pogodbo o obrestni meri
    \end{itemize}
\end{frame}

\begin{frame}
    \frametitle{Opcije na reguliranem trgu - opcije na terminsko pogodbo o obrestni meri}
    \begin{itemize}
        \item Nakupna opcija daje nosilcu pravico zavzeti dolgo pozicijo, prodajna pa kratko
        \item Cena terminske pogodbe je ob izvršitvi poravnana z izvršilno, nato pa se preveri tržno ceno
    \end{itemize}
\end{frame}

\begin{frame}
    \frametitle{Opcije na reguliranem trgu, 2. del}
        \begin{itemize}
            \item Oglejmo si naslednji zgled:
            \item Kupimo nakupno opcijo, terminska pogodba ima izvršilno ceno 85\textdollar
            \item Naj bo trenutna tržna cena te terminske pogodbe 95\textdollar, opcijo izvršimo
            \item Sedaj držimo dolgo pozicijo v terminski pogodbi s tržno ceno 95\textdollar, za katero smo plačali 85\textdollar
        \end{itemize}
    \end{itemize}
\end{frame}

\begin{frame}    
    \frametitle{Opcije na prostem trgu}
    \begin{itemize}
        \item Z OTC opcijami trgujejo investitorji, ki se želijo zaščititi pred specifičnimi tveganji
        \item Ginnie Mae opcije, posebne državne menice,\ldots
        \item Kupec tvega, da prodajalec ne bo mogel izplačati obveznosti
    \end{itemize}
\end{frame}


\end{document}
