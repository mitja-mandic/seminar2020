\documentclass[14pt]{beamer}
\usepackage[T1]{fontenc}
\usepackage[utf8]{inputenc}
\usepackage[slovene]{babel}
\usepackage{pgfpages} % privat zapiski
\usepackage{amsmath} % pravilen izpis v "math mode"
\usepackage{hyperref}
\hypersetup{hidelinks}


\usepackage{graphicx}% http://ctan.org/pkg/graphicx
\usepackage{booktabs}% http://ctan.org/pkg/booktabs

\usepackage{palatino}
\usefonttheme{serif}

\setbeamertemplate{navigation symbols}{} % izklop navigacije
\setbeamertemplate{footline}[frame number]{} % oštevilčenje
\setbeamertemplate{note page}{\pagecolor{yellow!5}\insertnote}
\setbeamertemplate{itemize items}[circle]

%\setbeameroption{hide notes}                        % samo prosojnice
%\setbeameroption{show only notes}                   % samo zapiski
%\setbeameroption{show notes on second screen=right}  % oboje
 
\begin{document}
\title{Opcije na obrestne mere \\ \textit{Interest Rate Options}}
\author{Mitja Mandić}
\date{21. april 2020}

\begin{frame}
    \titlepage
\end{frame}

\begin{frame}
    \frametitle{Uvod}
    Opcijam podobne pogodbe poznamo že iz antike
\end{frame}

\begin{frame}
    \frametitle{Na splošno o opcijah}
    \begin{itemize}
        \item Kupec dobi \textbf{pravico}, da od prodajalca kupi 
        ali mu proda osnovno premoženje
        \item Glede na možnost izvedbe poznamo tri osnovne skupine opcij:
        \pause
        \begin{itemize}
            \item evropske
            \pause
            \item ameriške
            \pause
            \item eksotične (modificirane ameriške, bermuda,\ldots)
        \end{itemize}
    \end{itemize}
\end{frame}

\begin{frame}
    \frametitle{Trgovanje z opcijami}
    \begin{itemize}
        \item Nakupne in prodajne opcije
        \item Ko nosilec plača premijo prodajalcu, nima več obveznosti, le še pravice
        \item Nosilec lahko izgubi največ premijo, prodajalec pa lahko največ toliko profitira
    \end{itemize}
\end{frame}

\begin{frame}
    \frametitle{Razlika med opcijami in terminskimi pogodbami}
    \begin{itemize}
        \item Nosilcu opcije po plačilu premije ni potrebno več reagirati
        \item Prodajalec opcije pa se mora ravnati po željah kupca - tu je kupec izpostavljen tveganju
        \item Drugačna shema tveganja in zaslužka
        \item Za nosilca ni potrebe po kritju, prodajalec pa kritje mora vzdrževati
    \end{itemize}
\end{frame}

\begin{frame}
    \frametitle{Opcije na reguliranem in prostem trgu}
    \begin{itemize}
        \item Opcije na borzi so standardizirane, posrednik pri trgovanju je klirinška hiša, transakcijski stroški so nižji
        \item OTC opcije bolj prilagodljive
        \item Na borzah prevladujejo opcije na terminsko pogodbo o obrestni meri
    \end{itemize}
\end{frame}

\begin{frame}
    \frametitle{Opcije na terminske pogodbe o obrestni meri}
    \begin{itemize}
        \item Osnovno premoženje je terminska pogodba o obrestni meri (\textit{futures option})
        \item Nakupna opcija daje nosilcu pravico zavzeti dolgo pozicijo, prodajna pa kratko
        \item Cena terminske pogodbe je ob izvršitvi poravnana z izvršilno, nato pa se preveri tržno ceno
    \end{itemize}
\end{frame}

\begin{frame}
    \frametitle{Opcije na terminske pogodbe o obrestni meri, 2. del}
        \begin{itemize}
            \item Oglejmo si naslednji zgled:
            \item Kupimo nakupno opcijo, terminska pogodba ima izvršilno ceno 85\textdollar
            \item Naj bo trenutna tržna cena te terminske pogodbe 95\textdollar, opcijo izvršimo
            \item Sedaj držimo dolgo pozicijo v terminski pogodbi s tržno ceno 95\textdollar, za katero smo plačali 85\textdollar
        \end{itemize}
\end{frame}

\begin{frame}
    \frametitle{Zakaj opcije na terminske pogodbe?}
    Trije glavni razlogi za prevlado opcij na terminske pogodbe:
    \begin{enumerate}
        \item ni potrebno plačevati natečenih obresti
        \item ni strahu pred pomanjkanjem terminskih pogodb in rastjo cen
        \item terminski pogodbi je lažje določiti ceno
    \end{enumerate}
\end{frame}
        
\begin{frame}    
    \frametitle{Opcije na prostem trgu}
    \begin{itemize}
        \item Z OTC opcijami trgujejo investitorji, ki se želijo zaščititi pred specifičnimi tveganji
        \item Opcije na Ginnie Mae obveznice, posebne državne menice, \ldots
        \item Kupec tvega, da prodajalec ne bo mogel izplačati obveznosti
        \item Manj likvidne, to načeloma ni problem
    \end{itemize}
\end{frame}

\begin{frame}
    \frametitle{Cena opcij}
    \begin{itemize}
        \item Osnovna vrednost (\textit{intrinsic value})
        \begin{itemize}
            \item $C_{t} = max\{\text{\textit{vrednost osnovnega premoženja}} - \text{\textit{izvršilna cena}}, 0\}$
            \item $P_{t} = max\{0, \text{\textit{izvršilna cena}} - \text{\textit{vrednost osnovnega premoženja}}\}$
        \end{itemize}
        \pause
            \item Časovna vrednost (\textit{time value})
            \begin{itemize}
                \item $max\{\textit{\text{cena opcije} - \text{osnovna vrednost}},0 \}$
                \item Če je izvršilna cena nakupne opcije 100\textdollar, tržna cena premoženja 105\textdollar, cena opcije 9\textdollar, je njena časovna vrednost 4\textdollar
            \end{itemize}
    \end{itemize}
\end{frame}

\begin{frame}
    \frametitle{Cena opcij}
        \resizebox{\linewidth}{!}{
        \begin{tabular} { c | c | c }
        
        & Nakupna & Prodajna \\
        \hline\hline
         Tržna cena > izvršilna cena & Tržna - Izvršilna & 0  \\ 
         & \textit{in-the-money} & \textit{out-of-the-money} \\
         \hline
         Tržna cena < izvršilna cena & 0 & Izvršilna - Tržna \\ 
         & \textit{out-of-the-money} &\textit{in-the-money}\\ 
         \hline
         Tržna cena = izvršilna cena & 0 & 0 \\
         & \textit{at-the-money} &  \textit{at-the-money}   
        
        
    \end{tabular}}
       
\end{frame}

%\begin{frame}
%\frametitle{Vplivi na ceno opcij}
%\resizebox{\linewidth}{!}{
%\begin{tabular}{l|c|c}
%    
%    Dejavnik & Nakupna opcija & Prodajna opcija \\
%    \hline\hline
%    Cena osnovnega premoženja & poveča & zmanjša \\
%    Izvršilna cena & zmanjša & poveča \\
%    Čas do zapadlosti & poveča & poveča \\
%    Pričakovane spremembe obrestnih mer & poveča & poveča \\
%    Kratkoročne netvegane obrestne mere & poveča & zmanjša \\
%    Kuponi & zmanjša & poveča
%
%\end{tabular}}
%\end{frame}

\begin{frame}
    \frametitle{Cena opcij na terminsko pogodbo}
    \begin{itemize}
        \item Cena opcije se navaja v 64-inah procenta nominalne vrednosti terminske pogodbe
        \item Cena 24 torej pomeni: $\frac{24/64}{100}\times 100.000\$ = 375\$$
    \end{itemize}
\end{frame}

\begin{frame}
    \frametitle{Kje se s takimi opcijami trguje?}
        \begin{itemize}
            \item \textit{Chicago Mercantile Exchange}:
                \begin{itemize}
                    \item Eurodollar opcije, ameriške državne obveznice, \ldots
                    \item lani 2.76 milijona pogodb na dan
                \end{itemize}
            \item \textit{NYSE Liffe}:
                \begin{itemize}
                    \item opcije se prodajajo kot terminski posli
                    \item kupec in prodajalec dnevno nalagata kritje
                    \item Euribor, britanski funt
                \end{itemize}
        \end{itemize}
\end{frame}

\begin{frame}
    \frametitle{Viri}
    \begin{itemize} 
        \item Frank. J. Fabozzi: \textit{Fixed Income Analysis}, John Wiley \& Sons $2$. izdaja, 2007
        \item Investopedia: \textit{Interest Rate Options} (leto ogleda: $2020$), \\
                dostopno na: \url{https://www.investopedia.com/terms/i/interestrateoptions.asp}
        \item thisMatter: \textit{Interest Rate Options} (leto ogleda: $2020$), \\
                dostopno na: \url{https://thismatter.com/money/options/interest-rate-options.htm}
    \end{itemize}
\end{frame}
\begin{frame}
    \begin{itemize}
        \item CME Group: \textit{Interest Rate Options}, (leto ogleda $2020$), \\
        dostopno na: \url{https://www.cmegroup.com/trading/interest-rates/options.html}
    \end{itemize}
\end{frame}

\end{document}
